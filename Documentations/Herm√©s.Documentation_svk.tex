\documentclass[a4paper]{article}

\usepackage[slovak]{babel}
\usepackage[utf8]{inputenc}
\usepackage[colorinlistoftodos]{todonotes}

\title{Projekt Hermés}

\author{Tomáš Krejčí \& Martin Mečiar}
\date{}




\begin{document}
\maketitle
\noindent\makebox[\linewidth]{\rule{\paperwidth}{0.4pt}}
\newpage
\section{Užívateľská dokumentácia}

Po spustení aplikácie sa nám otvorí okno Hermés demo aplication, v ktorom je potrebné vykonať následné kroky. 
\begin{enumerate}
\item Vyberieme typ Portfolia, ktoré budeme chcieť simulovať.
\item Vyberieme typ Brokera, ktorý bude uskutočnovať príkazy nákupu a predaju.
\item Vyberieme DataFeedy, ktoré z ktorých budeme čerpať dáta pre simuláciu. DataFeedy pridávame tlačítkom Add DataFeed. Po stlačení Add DataFeed sa nám otvorí nové okno, kde vyberieme DataFeed a postupne odzhora dolu budeme vyplňovať  jednotlivé políčka (všetky ukázané políčka sú pre daný typ DataFeedu povinné). Odstrániť DataFeed môžeme pomocou tlačítka Remove DataFeed.
\item Do vyplníme vstupný kapitál simulácie.
\item Spustíme simuláciu pomocou tlačítka Construct and initialize simulation.
\item Control flow simulácie môžeme ovládať z posledného riadku.
\end{enumerate}

\section{Programátorská dokumentácia}

\subsection{Z pohľadu projektov}
Hermés sa skladá z 2 projektov: Hermés.Core a Hermés.GUI. 

Hermés.Core obsahuje vnútornú implementáciu všetkých potrebných častí projektu, ktoré sa netýkajú grafického rozhrania. Je to čisto event-driven knižnica.

Hermés.GUI obsahuje implementáciu grafického rozhrania k projektu Hermés.Core. Je to demonstračná aplikácia využívajúca knižnicu Hermés.Core.


\subsection{Hermés.Core z pohľadu návrhu}
Základnou jednotkou simulácie je abstraktná trieda \textbf{Event}. Objekty medzi sebou komunikujú na základe eventov. Eventy v simulácií(všetky musia byť potomkami Eventu) sa delia podľa účelu na: \textit{MarketEventy}, \textit{SignalEventy}, \textit{OrderEventy}, \textit{FillEventy}. 

Základnými objektami pre simuláciu sú potomkovia/implementácie/instancie nasledovných objektov/interfacov: Kernel, DataFeed, IBroker, IStrategy, Portfolio.

Hlavným objektom simulácie je \textbf{Kernel}. Kernel vlastní čas, v ktorom sa momentálne situácia nachádza, prioritnú frontu eventov pre celú simuláciu a list objektov implementujúcich IEventConsumer(IBroker, IStrategy, Portfolio). Hlavný dispatcher eventov sa nachádza v tejto triede. V simulácií sa nachádza len jedna instancia Kernelu. Prístup ku Kernelu je jedine skrz Portfolio a má na sebe implementovanú metódu AddEvent. Pri dispatchovaní eventu je event ponúknutý všetkým IEventConsumerom.

V simulácií sa môže nachádzať 0 až N datafeedov dediacich od abstraktnej triedy \textbf{DataFeed}, z ktorých simulácia čerpá vstupné data z externého zdroja a mení ich na podobu eventov(\textit{MarketEventov}). Eventov z daného zdroja môže byť ľubovoľné množstvo. Dôležité je, aby pracoval na agregovaných OHLC dátach a nie na tick-by-tick dátach. Datafeedy môžu byť dva druhy: online(je to neustále streamovanie nových eventov – napr reálna burza) a offline(jednorázové historické dáta).

\textbf{Stratégie} určujú, podľa daného vstupu(väčšinou len podľa \textit{MarketEventov}), či je rozumné v danej chvíli akcie daného typu nakúpiť, predať alebo nič nevykonať a vytvára tak \textit{SignalEventy}. 

\textbf{Portfolio} riadi simuláciu a taktiež určuje z množiny stratégií, ktoré stratégie(konkrétne \textit{SignalEventy}) sa budú obchodovať. Rozhoduje sa podľa vnútornej logiky riadenej position sizingom a risk managementom. Príkazy, ktoré sa rozhodne uskutočňovať(konkrétne \textit{SignalEventy}) označí pre Brokera na vykonanie(zmení \textit{SignalEvent} na \textit{OrderEvent}). Pre danú simuláciu je vždy len jedno Portfólio. Vytvorí si automaticky instanciu Kernelu, ktorý je tiež len jeden pre simuláciu.

\textbf{Broker} spracuváva eventy predaja a nákupu položiek trhu vyhodnotených Portfoliom ako správny príkaz(\textit{OrderEventy}). Po skutočnom alebo simulovanom úspešnom uskutočnení príkazu vytvorí \textit{FillEvent}, ktorý si odoberie Portfolio. Broker je práve jeden pre danú simuláciu.


\subsection{Workflow pre Hermés.Core}
\begin{enumerate}
\item Vytvoríme inštanciu potomka triedy Portfolio.
\item Pridáme Brokera.
\item Zaregistrujeme DataFeedy.
\item Pridáme využívané stratégie.
\item Na Portfóliu zavoláme metódu Initialize, čím znemožníme zmenu vnútorného stavu(predošlých nastavení).
\item Na dispatcheri spustíme EventLoop.
\item Vo chvíli, keď chceme simuláciu zastaviť tak zavoláme StopProcesing.
\item Výsledky simulácie môžeme z niečoho vytiahnuť(terajšia verzia poskytuje prístup k dátam jedine cez Portfolio).
\end{enumerate}


\subsection{Portfolio.GetPortfolioValue}
Táto metóda na základe predošlých FillEventov spočíta na aktuálnu hodnotu Portfólia. Hodnota zahrňuje Cash+AktuálnaHodnotaVšetkýchPozícií.



\end{document}